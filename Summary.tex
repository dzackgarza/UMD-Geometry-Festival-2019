\input{/home/zack/Notes/Latex/preamble.tex}

\title{
	\textbf{Title}\\
	{\normalsize (subtitle)}
}
\author{D. Zack Garza}
\date{\today}

\begin{document}

\maketitle\todo{Insert title and subtitle.}
\tableofcontents


\hypertarget{umd-geometry-festival}{%
\section{UMD Geometry Festival}\label{umd-geometry-festival}}

\hypertarget{fluid-mechanics-and-geometry-yann-brenier}{%
\section{Fluid Mechanics and Geometry (Yann
Brenier)}\label{fluid-mechanics-and-geometry-yann-brenier}}

Outline:

\begin{itemize}
\tightlist
\item
  Geometric interpretation of Euler equations for incompressible fluids
  (Arnold 1966)
\item
  Discrete fluids, combinatorial optimization, generalized
  incompressible flows
\item
  Generalized minimizing geodesics with probability and convexity tools
  (1989-2012)
\item
  The initial value problem, relates to the completeness of certain
  manifolds (2018)
\end{itemize}

\hypertarget{euler-equations}{%
\subsection{Euler Equations}\label{euler-equations}}

First PDEs written in modern style: Euler, 1757.

The Euler flow is an incompressible fluid confined to a compact
Riemannian manifolds, evolving according to the Euler equations. (It
wasn't clear how to extend definition of Riemannian manifolds to
infinite dimensions.) Evolve along geodesics, need to be
volume-preserving. Look at action on gridded-up version of torus
\(T^2\).

Discrete version of incompressible motion - acts by permutation, i.e.~a
finite series of permutations on gridded cells. Compute ``cost'' of flow
by adding squares of displacements, which increases if any step moves
any given grid by a large amount.

There is some solution that gives the final permutation at the lowest
cost -- the possibilities are finite, but large. This will be our
``geodesic'', i.e.~optimal transport.

Example: find the minimal flow between \([1,2,\cdots n]\) and
\([n, n-1, \cdots, 2, 1]\). Transpositions get it done in 12 steps. This
is nice because it passes to the limit nicely, which ends up looking
like a solution to Euler's equation. Observe pattern of each number
``bouncing'' off wall, while its neighboring number does so in a
symmetric fashion the yields a rectangle.

Well-known: in 1D, there is no non-trivial solution.

In passing to the continuous limit, this solution can be improved by a
factor of \(\frac \pi {12}\). Actual solution in continuous case is
known; all trigonometric functions. This shape is reflected in the
finite numeric simulations.

This generalized solution has been discovered in an entirely different
framework involving random walks on the symmetric group (Virag et al).
The least action principle ruling the original problem is the limit of a
large deviation principle ruling the random exchange of adjacent cells
in the discrete model. This principle says that looking at Brownian
motion to a point, conditioned on the motion being in a neighborhood of
that point, converges to a geodesic in the limit as the noise goes to
zero.

\hypertarget{generalized-flows}{%
\subsection{Generalized Flows}\label{generalized-flows}}

A generalized incompressible flow on a compact Riemannian manifold \(D\)
is a probability measure \(\mu\) on paths \(\xi_t\) such that \(\mu\)
has finite average energy

\[
E_\mu \int_0^T \frac 1 2 \abs{\frac{d\xi}{dt}}^2 dt
\]

where we are taking the expectation of this integral.

Main results on minimizing geodesic (since 1992): let \(\mu_{0, T}\) be
a probability measure on \(D \times D\) such that the projections
\(\mu_0=\mu_t = \mathcal{L}_D\), the Lebesgue measure on \(D\). This
measure is spanned by at least on generalized incompressible flow
\(\mu\) of minimal energy.

In fact, there is a unique pressure distribution \(\nabla p(t, x)\) that
relates to these solutions by an explicit equation, where we view this
as an acceleration field. The uniqueness here is surprising. In Arnold's
classical framework, approximate minimizing geodesics may not converge
in the any classical sense but do converge to generalized solutions when
the dimension is at least 3 \emph{(Shnirelman 1985)}. In dimension 2,
there is something to do with symplectic forms that prevents this. Note
that there is no similar results for the dynamics of rigid bodies,
i.e.~geodesic curves along \(SO(3)\) \emph{(Y.B. 2012)}. Seems that the
\(d=2\) case is generally open.

Smoothness of these solutions is important.The existence of a unique
\(\nabla p\) follows from the convexity of the minimizing geodesic
problem. Main open equation: is \(\nabla p\) smooth? We currently know
it is \(L^2\) locally (1999/2008) and we have data for which \(p\) is
not better than locally semi-convex in \(x\).

Last questions: what about the initial value problem? (Y.B. 2018)

A priori, convex minimization techniques are hopeless for the IVP. For a
generalized incompressible flow \(\mu\) with finite average energy, it
does not make sense to prescribe any initial velocity
\(\left| \frac{d\xi}{dt}\right|_{t=0}\) for \(\mu-\)a.e. paths. But if
we look at such a flow in minimal energy, we can set up a different
equation that is tractable. The dual convex minimization problem is
always solvable, and can uniquely recover the smooth classical solutions
to the Euler equations for a short enough \(T\). This can be seen as a
kind of non-commutative optimal transport problem involving fields on
non-negative symmetric matrices, which are of current interest.

We currently don't know if the Euler equations have global solutions, or
even if they break down in finite time! Wide open, one of the main
problems in non-linear PDEs.

Generally looking at space of volume-preserving diffeomorphisms, dense
in the set of certain product measures? Gives a natural weak closure:
goes by the name of bi-stochastic measures?

\hypertarget{moment-maps-in-symplectic-and-kahler-geometry-dietmar-salamon}{%
\section{Moment Maps in Symplectic and Kahler Geometry (Dietmar
Salamon)}\label{moment-maps-in-symplectic-and-kahler-geometry-dietmar-salamon}}

Let \((X, \omega)\) be a symplectic manifold acted on by a lie group
\(G\). There is a moment map \(\mu: X \to \mathcal{Of}^*\) (?), and we
can consider a quotient space \(X//G = \mu^{-1}(0) / G\) where we look
at where the moment map vanishes and quotient out by a group action.
This is an orbifold, and in fact a manifold and symplectic.

We can look at the Kahler case \((X, \mu, J)\) and for \(G\in U(n)\) we
can complexify to obtain \(G^C\in G(n, \mathbb C)\). and
\(X/G^C \cong X^{ss}/\sim \cong X//G\). We'd like to know the behavior
of maps passing through the zero set of the moment map. To this end, we
use Mumford weights: \[
W_\mu(x, \xi) = \lim_{t\to\infty}\left< \mu (\exp(it\xi)x), \xi\right>
\] where we follow a Hamiltonian flow to infinity and obtain a number.

If any of the weights are negative, the points can not be semistable,
which falls out of a specific bound on the Mumford weights. Thus we have

\textbf{Theorem}: \[
\sup_{\xi\neq 0} \frac{-W_\mu(x,\xi)}{\left|\xi\right|} \leq \inf_{g\in G} \left|\mu(gx)\right|
\]

\textbf{Corollary}: \(x\) is semistable (\(\mu(gx) = 0\)) \(\iff\)
\(\forall\xi, W_\mu(x, \xi) \geq 0\).

This is at the heart of \textbf{geometric invariant theory} (Hilbert,
Mumford, Kempf, Ness, Kirwan).

The key tool is studying the gradient flow of the moment map, or more
precisely \(\frac 1 2 \mu^2\), about which one can say many interesting
things.

\textbf{Example}: let \[
J_n = \left\{ J^{2n\times 2n} \mid J^2 = -I \right\}
= \left\{ g \begin{pmatrix}0 & -I \\ I & 0\end{pmatrix} g \mid g\in SL(2n, \mathbb R)\right\}
\] which are compatible with orientation. We can take the tangent space
to obtain \[
T = \left\{ J \mid \hat J J + J\hat J = 0\right\} = \left\{ [\xi, \hat J] \mid \xi \in \mathfrak{sl}(2, \mathbb R)\right\},
\] and get a symplectic form
\(\Omega_J(A, B) = \frac 1 2 \mathrm{Trace}(AJB)\). Then the moment map
is given by \[
\Omega_J([\xi, J], \hat J) = \left< d\mu(), (J)\hat J, \xi\right> = -\mathrm{Trace}(\xi\hat J).
\]

Can look at frame bundle, get structure group \(SL(2n, \mathbb R)\). Let
\(M^{2n}\) be an oriented manifold with a volume form, then \(J(M)\) is
the space of almost-complex structures on \(M\). Can repeat the above
construction to get a symplectic form that is volume-preserving by
integrating a differential form over the endomorphism bundle. This ends
up being compatible with the group action (the group of
volume-preserving diffeomorphism), so the question becomes whether or
not it's a Hamiltonian action.

We want to get a map that yields a exact 2-form, we'll just show one
that yields a closed form -- the Ricci form (?).

\textbf{Theorem}: The action of \(G^{Ox}\) on \(J(M)\) is Hamiltonian
with moment map \(J(M) \to \Omega^2(M)\) given by
\(J \mapsto 2\mathrm{Ricc}_{G, J}\).

Choose \(\nabla\) a torsion-free connection on \(TM\) where
\(\nabla_G = 0\), then
\(\mathrm{Ricc}_{G, J} = \frac 1 2 (\tau^\nabla_J + d\lambda^\nabla_J)\)
where \(d\) is the covariant derivative. Although we make this choice,
the result Ricci form does not depend on it.

Think about the case where \(M\) is Calibi-Yau (and/or Teichmüller
space?). Can do similar construction, but the Ricci form takes values in
some affine space and needs a correction, and you lose equivariance.

Note that this is not a Kahler form. Look at Yau's theorem. Look at the
Chern class of \(S^1\times S^3\). There is a construction that yields
both a complex form and a canonical symplectic form on the Teichmüller
space \(\mathcal T_0(M)\), and we can find an explicit formula for it:
\[
\Omega_J(\hat J_1, \hat J_2) = \int_M \left(\frac 1 2 \mathrm{tr}(\hat J_1 J \hat J_2) - f_1g_2 + f_2g_1 \right)\rho_J
\] where \(\rho_J\) is such that \(\mathrm{Ricc}_{\rho_J, J} = \omega\).

Recovers something about Kahler-Einstein metric, the moment map can also
be roughly seen as a Kahler-Ricci potential. Yields some logarithmic
variant of the Kahler-Ricci flow. See the Dehn functional, Donaldson
framework.

\hypertarget{zero-sets-of-laplace-eigenfunctions-aleksandr-logunov}{%
\section{Zero Sets of Laplace Eigenfunctions (Aleksandr
Logunov)}\label{zero-sets-of-laplace-eigenfunctions-aleksandr-logunov}}

Let \(M\) be a closed Riemannian manifold, we will look at
eigenfunctions of the Laplacian on on \(M\).

Example: Eigenfunctions on \(S^2\) are restrictions of homogeneous
harmonic polynomials functions \(f: \RR^3 \to S^2\), which has a basis
of relatively simple polynomials where the eigenvalues are related to
their degrees.

Two elementary questions:

\begin{itemize}
\item
  Does the number of critical points of eigenfunctions
  \(\varphi_\lambda\), \[
  C_{\phi, \lambda} = \theset{x \suchthat \nabla \varphi_\lambda(x)= 0},
  \] tend to infinity as \(\lambda \to \infty\)?
\item
  Flat eigenfunctions: is there a sequence of eigenfunctions on \(S^2\)
  such that \[
  \max_M \abs{\varphi_\lambda(x)} \leq C \norm{\varphi_\lambda}_2?
  \]

  \begin{itemize}
  \tightlist
  \item
    E.g. on \(S^1\), all eigenfunctions \(\sin(ax+b)\) are flat.
  \item
    Such eigenfunctions exist on \(S^{2d-1}\).
  \item
    Sarnak's conjecture: there are no such sequences on \(S^2\).
  \end{itemize}
\end{itemize}

Nodal domains: a theorem of Courant (1923) says that the \(k\dash\)th
eigenfunction has at most \(k\) nodal domains (where they are ordered by
size of eigenvalue). A result of Stern/Lewey shows that there are
spherical harmonics of any odd degree with only two nodal domains.

Yau's conjecture: \[
c\sqrt\lambda \leq H^{n-1}(Z_{\varphi_\lambda}) \leq C \sqrt \lambda.
\]

Shown for algebraic manifolds and real-analytic manifolds, still open in
the Riemannian case. Recent result: proves lower bound in dimensions
above 2, some improvements closer to conjectured in dimension 2.

Nadirashvili's conjecture: Let \(u\) be a non-constant harmonic function
in \(\RR^3\). Does \(\mu(\theset{u=0}) = 0\)? Believed we need to
understand this for harmonic functions before understanding the zero
sets of Laplacians. By 2016 work, yes, and there is a uniform lower
bound (?).

An old trick: can translate questions about zero sets of eigenfunctions
\(\varphi + \lambda \varphi = 0\) into zero sets of harmonic functions
\(\Delta u = 0\), given by defining
\(u(x,t) = \varphi(x)e^{\sqrt\lambda t}\).

Work of Donnelly-Fefferman bounds the growth of Laplace eigenfunctions.
There is a harmonic analog of Yau's conjecture,
\(H^{n-1}(Z_\varphi \intersect B_1) \leq CN(B_1)\) where \(H\) is the
Hausdorff measure and \(N\) is the doubling index; Yau's would follow
from this by setting \(C=\sqrt\lambda\).

On the scale \(c/\sqrt \lambda\), you can ``shake'' the eigenfunctions
so they look like harmonic functions; i.e.~there is a quasi-conformal
change of coordinates that sends the zero set the eigenfunctions to the
zero set of a harmonic function.

Landis conjecture: let \(\Delta u + Vu = 0\) be an elliptic equation,
where \(V\) is a bounded potential \(\abs V < 1\). If
\(\abs {u(x)} \leq\exp(-\abs{x}^{1+\varepsilon})\), then \(u\) is
identically zero. Can also be formulated in terms of the maximum number
of nodal curves intersecting at a point. WIP: this is true for real
potentials. Solutions behave very differently between the real and
complex cases. Any counterexample would necessarily have many zeros.

\hypertarget{the-geometry-and-arithmetic-of-the-worlds-smallest-calibi-yau-threefolds-jim-bryan}{%
\section{The Geometry and Arithmetic of the World's Smallest Calibi-Yau
Threefolds (Jim
Bryan)}\label{the-geometry-and-arithmetic-of-the-worlds-smallest-calibi-yau-threefolds-jim-bryan}}

\textbf{Definition}: A compact complex Kahler manifold \(X\) of
dimension \(d\) is Calibi-Yau if \(c_1(TX)=0\), where \(c_1\) is the
first Chern class (i.e.~\(\det T^\dual X = K_X \cong X\cross\CC\),
i.e.~the canonical is trivial) \(\iff\) there exists a holomorphic
global \(d\dash\)form \(\iff\) there exists a Ricci flat Kahler metric
and \(h^{k, 0}(X) = 0\) unless \(k=0,d\).

\begin{itemize}
\tightlist
\item
  If \(d=1\), \(X\) is an elliptic curve, topologically only one type (a
  torus)
\item
  If \(d=2\), \(X\) is a \(K3\) surface, topologically only one type
\item
  If \(d=3\), \(X\) is a CY-threefold, there are \textgreater5 million
\end{itemize}

There are two interesting Hodge numbers: - \(h^{1,1}(X)\): the dimension
of the Kahler cone, or dually the number of independent holomorphic
curve classes - Never zero if Kahler - \(h^{2,1}(X)\): the space of
independent infinitesimal complex deformations - Mirror symmetry swaps
these two

\textbf{Definition}: A rigid CY3 \(X\) has \(h^{2,1}(X)=0\).

Interested in a few aspects of CY3s: - What combinations of values
\((h^{1,1}, h^{2,1})\) are possible? - Mirror symmetry falls out of
tabulating these - Physicists interested in small ones - How small can
these combinations be? (By summing these two numbers) - Can we compute
the Donaldson-Thomas/Gromov-Witten/``string partition functions''
(generating functions for counts of curves) - What are the modular
properties of these functions? - Arithmetic: rigid CY3s over \(\QQ\) are
modular forms (much like Wile's theorem showing elliptic curves are
modular)

Today: - Construct rigid CY3s (banana manifolds) of (0,4) and (0,2) -
Compute Donaldson-Thomas partition function and associated Gromov-Witten
potential, a genus 2 Siegal modular form of weight \(2g-2\). - Compute
the modular forms (\(L\dash\)series)

\hypertarget{the-generic-banana-manifold}{%
\section{The Generic Banana
Manifold}\label{the-generic-banana-manifold}}

Start with a generic hypersurface \(S \subset \CP^2\cross \CP^1\) of
degree \((3,1)\). Projecting onto \(\PP^2\) yields
\(Bl_{9\text{pts}}\CP^2\), onto \(\PP^1\) yields an elliptic fibration
(projects onto an elliptic curve) which has 12 nodal fibers.

See image

Now to make a threefold: take the fiber product
\(\hat X = S \cross_{\PP^1} S\) which has 12 conifold singularities,
which we can resolve by taking \(X = Bl_\Delta(\hat X)\), where
\(\Delta\) is the diagonal. This yields a CY3 with
\(h^{2,1}(X) = 8, h^{1,1}(X) = 20\). The fibers are generically products
of elliptic curves \(E\cross E\), which has 12 singular fibers.

(Note that we're blowing up along a Cartier divisor and not a Weil
divisor, which gives the singularities local resolutions.)

We also have \(E_\text{sing} = \CC{\units}\cross \CC\units \cross B\)
where \(B=C^1 \union C^2 \union C^3\) where \(C_i = \PP^1\) and
\(C_i \intersect C_j = \theset{p,q}\) and \(N = \OO(-1)\oplus \OO(-1)\).
This thing is a toric variety?

See image

We can do enumerative geometry here by looking at the fibers of
\(X \mapsvia{\pi} \PP^1\), where
\(\beta \in \ker \pi \cong \ZZ^3 = H_2(X; \ZZ)\) is given by
\(\beta = d_1[c_1] + d_2[c_2] + d_3[c_3]\). Can write out a generating
function \(Z^{DT}(Q_1, Q_2, Q_3, p)\) where the coefficients are the DT
invariants, i.e.~the virtual count of curves given by the holomorphic
euler characteristic. This expands to a product where the coefficients
and exponents are themselves coefficients of modular forms. The
exponents appearing are the fourier coefficients of a specific modular
form?

The associated GW potentials are genus 2 Siegel modular forms \(F\) of
weight \(2g-2\). Can think of this as a function on the moduli space of
genus 2 modular curves, and the weight is like the Chern class?

A rigid example: start with an extremal elliptic surface \(S^6\), see
image. This is the universal family for moduli of elliptic curves with a
6-torsion point, and has a \(\ZZ/6\ZZ\) action. Let \(S'\) be a base
change that reverses the order of the fibers, and take
\(S\times_{\PP^1}S'\). The \(\ZZ/6\ZZ\) action is now free and this is a
\(6\times 4\) conifold. If we blow up the quotient along these points to
get an \(X_6\), this is a rigid CY3 where \(X\mapsvia{\pi} \PP^1\) has 3
banana fibers and 1 section, \(h^{2,1}=0\), and \(h^{1,1} = 4\). The
theorem (in progress) is that \(Z\) of these fibers is similar to the
previous one, and the \(F_q\)s are now Siegel for certain congruent
subgroups of \(\mathrm{Sp}_2(\ZZ)\).

As it turns out, there was an additional \(\ZZ/2\ZZ\) action, yielding a
\(\ZZ/12\ZZ\), and quotienting by this yields 2 banana fibers and 2
doubled fibers which are each \(\PP^1\) bundles over a special elliptic
curve. Moreover, this is tied for the smallest one, at
\(h^{2,1} = 0, h^{1,1} = 2\).

Where does the modularity come from? Topological vertex, Schur
functions, but not satisfying -- we don't have good explanations of
where this should be coming from.

There aren't any threefolds where we can completely compute the
partition functions, we have a hard time computing with the sections
(although the fibers are okay).

\hypertarget{boundary-operator-associated-to-sigma_k-curvature-yi-wang}{%
\section{Boundary Operator Associated to Sigma\_k Curvature (Yi
Wang)}\label{boundary-operator-associated-to-sigma_k-curvature-yi-wang}}

Define the \(k\)-Hessian energy \[
\int_\Omega u \sigma_k D^2 u ~dx.
\]

When \(k=1\) this recovers the Dirichlet energy
\(\int_\Omega -u\Delta u\). On a Riemannian manifold, we are instead
interested in the Shouten tensor which is a combination of the Ricci
curvature and the scalar curvature. Can decompose the Riemann curvature
tensor into a conformally invariant Weyl curvature \(W\) and and certain
product.

We want to look at conformally invariant operators, and the most natural
one is the conformal Laplacian \(L_g = -\Delta +c_n R_g\). Under a
conformal change of metric, we can look at the Yamabe problem which asks
for a decomposition with \(\hat R\) constant. This was solved in the
80s. It is a variational problem.

On 4-manifolds, there is a Chern-Gauss-Bonnet formula, which involves
the \(L_2\) norm of the Weyl tensor (local conformal invariant) and the
other is a constant times \(\sigma_2\), and so \[
\int_{X^4} \sigma_2 ~dv_g
\] is a conformal invariant.

This leads to a generalized Yamabe problem
\(\sigma_k = \text{constant}\). This is 2nd order elliptic PDE. This
problem is variational (solutions are critical points of an energy
functional) if \(k=1,2\) or \(g\) is locally conformally flat. Can write
this functional, \(\int \sigma_k ~\mathrm{vol}_g\), and can write an
Euler-Lagrange equation for this to determine the gradient of this
functional.

Open questions: - Is there a Dirichlet principle for the Shouten tensor
or the operator \(\sigma_k(D^2u)\)?

Why should this be possible? The \(k\dash\)Hessian energy is pointwise
non-negative when \(u=0\) on the boundary, so it is a generalized
Dirichlet energy by integration by parts. Can obtain a Sobolev
inequality for it when \(\Sigma\) is \((k-1)\)-convex (i.e.~this holds
for the second fundamental form) and \(2k<n\). This gives an embedding
into some \(L^p\) space. For \(2k=n\) obtain an Orlicz-type inequality;
for \(2k>n\) obtain embeddings into Holder spaces.

The well-known Dirichlet principle: for any \(u\) where
\(\restrictionof{u}{\del M} = f\) and
\(\int_\Omega \abs{\nabla u}^2 dx \geq \int_\Omega \abs{u_f}^2 dx\)
where \(u_f\) is a harmonic extension of \(f\) that solves \[
\begin{cases}
\Delta u = 0 & \text{on}~\Omega, \\
u=f & \text{on}~\del\Omega.
\end{cases}
\]

The main result is that for smooth domain with boundary, there is a
multilinear differential operator along with a multilinear functional
that is symmetric in its inputs that recovers the \(k\dash\)Hessian.
This functional looks like \[
Q = -\int_\Omega u L_k + \oint_{\del \Omega} B_k
\] where \(L_k\) is a \(k\dash\)linear map and \(B_k\) is a boundary
operator.

Examples involve the Laplace-Beltrami operator.

The symmetric property in this theorem yields two nice properties: the
functional's Euler-Lagrange equation is the one wanted earlier, and it
is convex. With some assumptions, this will yield the desired Dirichlet
principle for the \(\sigma_k\) curvature.

The boundary operator involves some generalization of the mean
curvature, called \(H_k\) which is a linear combination of the Shouten
tensor and the second fundamental form. The \(L\) is an extension of the
conformal Laplacian operator. When \(k=1\), this recover Escoba Yamabe's
functional.

\hypertarget{spectral-asymptotics-on-stationary-spacetimes-steven-zelditch}{%
\section{Spectral Asymptotics on Stationary Spacetimes (Steven
Zelditch)}\label{spectral-asymptotics-on-stationary-spacetimes-steven-zelditch}}

Want to look at a step function that jumps at eigenvalues by the
multiplicity of each eigenvalues. Recall the wave equation
\(\square u = 0\), and introduce the propagator
\(e^{-it\sqrt{-\Delta}}\) which is a pseudo-differential operator that
propagates time zero solutions.

Can look at the tempered distribution \[
\Tr V(t) = \sum_{j=0}^\infty e^{it\lambda_j} = e_0(t) + \sum_{L} e_L(t)
\]

were \(L\) are taken in the ``length spectrum''.

The singular support of \(e_0\) is \(\theset{0}\) and of \(e_L\) is
\(\theset{L}\). Essentially amounts to take the fourier transform of the
counting function and looking at its singularities. Need to assume that
the sets of closed geodesics are nondegenerate (critical point for the
length function on the loop space). Note that when you look at the
family of closed geodesics of a given length, the number will depend on
the geometry. A sphere has a manifold's worth, a torus a single
parameter family, while on a hyperbolic surface each length is isolated.

Any closed geodesic is in the unit cotangent bundle \(S\Sigma\), so pick
a transversal. Can look at first return map, take the derivative to
obtain the linear Poincare map. Shouldn't have 1 as an eigenvalue,
otherwise you could deform the geodesic (condition is equivalent to
being nondegenerate).

Major milestone: able to reduce complex determinants of Hessians of
phase functions to (essentially) linear algebra. Some kind of ``symbol
calculus'' of fourier operators.

General question: instead of solving Einstein's field equations, how do
we instead let a wave equation evolve on a curved spacetime?

Uncurved spacetime is given by \(\RR \cross \Sigma\) with a metric
\(-dt^2 + h_\Sigma\), the Euclidean metric? The generalization here is
to globally hypberolic stationary spacetimes, want a compact Cauchy
hypersurface. Globally hyperbolic is a condition used frequently in
general relativity. We have \(M^{3,1} \cong \RR \cross \Sigma\)
topologically (not metrically), and so every causal curves (tangent
vector has lorentz norm strictly positive?) intersects a given
hypersurface \(\Sigma\) exactly once.

Moreover \(\square\) is essentially the Laplace-Beltrami operator on
this space. Stationary means there exists a timelike Killing vector
field, i.e.~timelike flows are isometries. The killing field is our
stand-in for \(\dd{}{t}\), since there is no preferred time coordinate.
This is a strong assumption, since the cosmological constant was found
to be positive (i.e.~expanding spacetime). What is the propagator, the
eigenvalues, the lengths of closed geodesics?

Can look at \(z = \dd{}{t}\) and \(Dz = \frac 1 i \dd{}{t}\) and we can
recover the Laplacian by looking at the eigenvalues of special
solutions, \(Du_J = \lambda u_j\), and the \(u_j\) span
\(\ker \square_g\) the Klein-Gordon operator. Want to keep everything
internal with Lorentzian geometry, so don't want to choose a particular
Cauchy hypersurface extrinsically.

Can look at space of null-geodesic
\(N = \theset{\gamma \suchthat g(\dot\gamma, \dot\gamma) = 0}\)
(equivalent to light rays) which is a symplectic manifold. Define
\(\mathrm{char}\square = \theset{\xi \in T^\dual M \suchthat \sigma_\square (x, \xi) = 0}\),
then let \(\RR\) act on this by Hamiltonian flows to obtain
\(N = \mathrm{char}\square/\RR\). Leads to studying \(e^{tZ}\) the
killing flow on \(N\). Can get a bundle of the original space over the
orbits of the killing flow, which has a natural connection \(\theta\)
where the metric on the base space is induced by the metric in the total
space. The metric does not need to be integrable; this is equivalent to
requiring the space to be static instead of stationary.

The theorem: look at the spectrum of
\(\restrictionof{D_z}{\ker\square}\). Want to define a trace, so need an
inner product on \(\ker\square\). This has a natural symplectic vector
space structure, since it's a solution to variational problem (?), so we
can just add an almost-complex structure (see books on QFT in curved
spacetime, positive and negative frequency solutions, Robert Wahl?). Can
use the energy-stress tensor to define such an inner product, this will
be independent of the choice of a cauchy hypersurface.

\textbf{Theorem}: \(\Tr ~\restrictionof{e^{tZ}}{\ker \square}\) is
essentially the same summation formula as earlier, where the sum is over
the periodic orbits of the killing flow, and this is something that
makes sense for any dynamical system.

Picture to keep in mind: can evolve a hypersurface using the wave
equation \(\Sigma_0 \to e^{tZ}\Sigma_0\) and then look at the killing
flow \(e^{tZ}\Sigma_0 \to \Sigma_0\). This gives an operator that you
can take the trace of.

\listoftodos

\bibliography{/home/zack/Notes/library.bib}

\end{document}
